\documentclass[aspectratio=169, 10pt]{beamer}

% --- Paquetes Fundamentales ---
\usepackage[utf8]{inputenc}
\usepackage[spanish]{babel}
\usepackage{tikz}
\usepackage{tcolorbox}
\tcbuselibrary{skins} 
\usepackage{fontawesome5} 
\usepackage{lmodern} 

% --- Configuración de Colores Premium (Paleta "Deep Ocean") ---
\definecolor{primary}{RGB}{0, 45, 98}      
\definecolor{secondary}{RGB}{0, 112, 192}  
\definecolor{accent}{RGB}{255, 165, 0}     
\definecolor{lightgray}{RGB}{240, 240, 240}
\definecolor{darkgray}{RGB}{50, 50, 50}

% --- Personalización del Tema ---
\setbeamercolor{structure}{fg=primary}
\setbeamercolor{normal text}{fg=darkgray}
\setbeamercolor{background canvas}{bg=white}

% Quitar barra de navegación
\setbeamertemplate{navigation symbols}{}

% --- Diseño de Fondo ---
\setbeamertemplate{background}{
    \begin{tikzpicture}[remember picture,overlay]
        \fill[primary] (current page.north west) rectangle ([xshift=0.3cm]current page.south west);
        \fill[secondary, opacity=0.8] (current page.north east) -- +(-2,0) -- +(0,-2) -- cycle;
        \fill[accent, opacity=0.1] ([xshift=-1cm, yshift=1cm]current page.south east) circle (2cm);
    \end{tikzpicture}
}

% --- Título de Diapositiva ---
\setbeamertemplate{frametitle}{
    \vspace{0.2cm}
    \begin{beamercolorbox}[wd=\paperwidth,ht=1.2cm,dp=0.5cm,leftskip=0.5cm]{frametitle}
        \Large \textbf{\insertframetitle} % Reducido de \Huge a \Large
        \ifx\insertframesubtitle\@empty\else
            \par\small\color{secondary}\insertframesubtitle
        \fi
    \end{beamercolorbox}
    % Eliminada la línea decorativa inferior para evitar desbordes
}

% --- Cajas Personalizadas ---
\newtcolorbox{premiumbox}[1]{
    enhanced, 
    colback=lightgray,
    colframe=primary,
    arc=5pt,
    boxrule=1pt,
    title=\textbf{#1},
    coltitle=white,
    fonttitle=\large,
    shadow={2mm}{-2mm}{0mm}{black!20}
}

\newtcolorbox{alertbox}[1]{
    enhanced, 
    colback=red!5,
    colframe=red!70!black,
    arc=5pt,
    boxrule=1pt,
    title=\textbf{#1},
    coltitle=white,
    fonttitle=\large,
    shadow={2mm}{-2mm}{0mm}{black!20}
}

% --- Metadatos ---
\title{\textbf{Agente de Navegación Autónoma}}
\subtitle{Implementación con Policy Gradient (REINFORCE)}
\author{\textbf{Taller de Machine Learning}} 
\date{\today}

\begin{document}

% --- Diapositiva 1: Portada ---
{
\setbeamertemplate{background}{
    \begin{tikzpicture}[remember picture,overlay]
        \shade[top color=primary, bottom color=secondary] (current page.south west) rectangle (current page.north east);
        \draw[white, opacity=0.1, line width=2pt] ([xshift=2cm]current page.south west) circle (5cm);
        \draw[white, opacity=0.1, line width=2pt] ([xshift=-2cm]current page.north east) circle (4cm);
        \draw[white, opacity=0.05, step=1cm] (current page.south west) grid (current page.north east);
    \end{tikzpicture}
}
\begin{frame}[plain]
    \centering
    \vspace{0.5cm}
    \color{white}
    \Huge \textbf{Agente de Navegación Autónoma} \\
    \vspace{0.3cm}
    \Large \textit{Optimizando Rutas con Inteligencia Artificial}
    \vspace{1.5cm}
    \begin{tcolorbox}[enhanced, colback=white!10, colframe=white, width=0.6\textwidth, arc=10pt, boxrule=0pt]
        \centering \color{white} \textbf{Actividad 2 - Policy Gradient}
    \end{tcolorbox}
    \vspace{1cm}
    \small \today
\end{frame}
}

% --- Diapositiva 2: El Desafío ---
\begin{frame}{El Desafío (Inteligencia)}
    \begin{columns}
        \column{0.6\textwidth}
        \begin{premiumbox}{Objetivo Principal}
            Desarrollar un agente capaz de navegar autónomamente en un entorno complejo (Grid 9x9) desde un punto \textbf{A} hasta un punto \textbf{B}.
        \end{premiumbox}
        \vspace{0.5cm}
        \textbf{\faExclamationTriangle \ Dificultades del Entorno:}
        \begin{itemize}
            \item \textbf{Obstáculos ("Hoyos"):} Terminan el episodio inmediatamente (Castigo -1.0).
            \item \textbf{Recompensas:} Meta principal (+1.0) y Objeto opcional (+0.7).
        \end{itemize}
        \column{0.35\textwidth}
        \centering
        \begin{figure}
            \centering
            \begin{tikzpicture}[scale=0.8]
                \draw[step=0.5cm, gray!30, very thin] (0,0) grid (3,3);
                \fill[green!50] (0.1,2.6) rectangle (0.4,2.9); % Start
                \fill[red!50] (1.6,1.6) rectangle (1.9,1.9);   % Hole
                \fill[blue!50] (2.6,0.1) rectangle (2.9,0.4);  % Goal
                \draw[->, thick, primary] (0.25, 2.75) -- (1.25, 2.75) -- (1.25, 1.25) -- (2.75, 1.25) -- (2.75, 0.25);
            \end{tikzpicture}
            \vspace{0.2cm}
            \caption{\textit{Esquema de Navegación}}
        \end{figure}
    \end{columns}
\end{frame}

% --- Diapositiva 3: La Estrategia ---
\begin{frame}{La Estrategia (Proceso Empresarial)}
    \begin{alertblock}{Cambio de Paradigma}
        Abandonamos la programación imperativa ("Si X, haz Y") por un enfoque de \textbf{Aprendizaje Automático}.
    \end{alertblock}
    \vspace{0.5cm}
    \begin{itemize}
        \item[\faBrain] \textbf{Aprendizaje Empírico:} El agente no conoce las reglas físicas; las descubre interactuando (Prueba y Error).
        \item[\faRedo] \textbf{Simulación Masiva:} Ejecución de \textbf{10,000 episodios} para refinar la política de toma de decisiones.
        \item[\faChartLine] \textbf{Optimización:} El sistema mejora gradualmente buscando maximizar la recompensa total acumulada.
    \end{itemize}
\end{frame}

% --- Diapositiva 4: Solución Técnica ---
\begin{frame}{Solución Técnica (Tecnología)}
    \begin{columns}
        \column{0.55\textwidth}
        \textbf{Algoritmo: Policy Gradient (REINFORCE)}
        \vspace{0.3cm}
        El "cerebro" del agente es una tabla de probabilidades (Softmax) que se ajusta dinámicamente.
        \vspace{0.3cm}
        \begin{premiumbox}{Mecánica de Aprendizaje}
            \begin{enumerate}
                \item El agente actúa aleatoriamente al inicio.
                \item Si gana $\rightarrow$ \textbf{Refuerza} las acciones.
                \item Si pierde $\rightarrow$ \textbf{Debilita} las acciones.
            \end{enumerate}
        \end{premiumbox}

        \column{0.4\textwidth}
        \centering
        % --- CORRECCIÓN 1: Color del texto forzado ---
        \begin{tcolorbox}[enhanced, colback=black!85, colframe=black, title=\textbf{Stack Tecnológico}, shadow={2mm}{-2mm}{0mm}{black!30}]
            % Forzamos el color blanco explícitamente dentro de la caja y para los items
            \color{white}
            \setbeamercolor{itemize item}{fg=white}
            \setbeamercolor{item}{fg=white}
            \setbeamercolor{normal text}{fg=white}
            \usebeamercolor[fg]{normal text}
            \begin{itemize}
                \item \textbf{\faPython \ Python 3.x}: Núcleo.
                \item \textbf{\faCubes \ NumPy}: Matemática.
                \item \textbf{\faChartBar \ Matplotlib}: Gráficos.
            \end{itemize}
        \end{tcolorbox}
    \end{columns}
\end{frame}

% --- Diapositiva 5: Resultados y Riesgos ---
\begin{frame}{Resultados y Riesgos (Proceso de Cambio)}
    \begin{premiumbox}{Hallazgo Clave: Aversión al Riesgo}
        El agente desarrolló un comportamiento \textbf{"Conservador"}.
    \end{premiumbox}

    \vspace{0.2cm}

    \begin{columns}
        \column{0.6\textwidth}
        \footnotesize 
        \begin{itemize}
            \item \textbf{Observación:} El agente ignora sistemáticamente el objeto de bonificación (+0.7).
            \item \textbf{Causa:} El objeto está rodeado de "hoyos". El riesgo de muerte (-1.0) supera la ganancia potencial.
            \item \textbf{Conclusión:} El algoritmo prioriza la seguridad y el cumplimiento de la misión principal.
        \end{itemize}
        
        \column{0.35\textwidth}
        \centering
        % --- CORRECCIÓN 2: 'transform shape' para arreglar el zoom ---
        \begin{tikzpicture}[scale=0.7, transform shape] 
            \node[circle, fill=green!20, draw=green, thick, minimum size=1.2cm] (safe) at (0,0) {\footnotesize Seguro};
            \node[circle, fill=red!20, draw=red, thick, minimum size=1.2cm] (risk) at (2.5,0) {\footnotesize Riesgo};
            \draw[->, very thick, primary] (safe) to[bend left] node[above] {\scriptsize Preferencia} (risk);
            \node at (1.25, -1) {\scriptsize \textbf{Decisión del Agente}};
        \end{tikzpicture}
    \end{columns}
\end{frame}

\end{document}
